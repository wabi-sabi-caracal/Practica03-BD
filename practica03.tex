\documentclass[letterpaper,11pt]{article}
% Soporte para los acentos.
\usepackage[utf8]{inputenc}
\usepackage[T1]{fontenc}
% Idioma español.
\usepackage[spanish,mexico, es-tabla]{babel}
% Soporte de símbolos adicionales (matemáticas)
\usepackage{multirow}
\usepackage{amsmath}
\usepackage{amssymb}
\usepackage{amsthm}
\usepackage{amsfonts}
\usepackage{latexsym}
\usepackage{enumerate}
\usepackage{ragged2e}
\usepackage{mathtools}
\usepackage{float}
% Soporte para referencias y citas
\usepackage{hyperref}
% Soporte para imágenes.
\usepackage{graphicx}
% Modificamos los márgenes del documento.
\usepackage[lmargin=2cm,rmargin=2cm,top=2cm,bottom=2cm]{geometry}

\title{Fundamentos de Bases de Datos \\
       Práctica 03. }

\author{Teresa Becerril Torres
        $\#$ de cuenta: $315045132$ \\
        Miguel Ángel Torres Sánchez
        $\#$ de cuenta: $315300442$ \\
        Nicole Romina Traschikoff García
        $\#$ de cuenta: $315164482$ \\
        Tania Michelle Rubí Rojas
        $\#$ de cuenta: $315121719$}

\date{09 de septiembre del 2019}
\begin{document}
\maketitle

\section{Reporte}
Para nuestro diseño de la base de datos tomamos las siguientes decisiones:

\textbf{- Entidades: }
\begin{itemize}
\item Como es necesario llevar el control de las tarjetas SIM
      creamos una entidad \textbf{SIM}, consideramos como llave
      primaria a \textbf{ICCID} (identificador de la SIM) y su
      único atributo será \textbf{Tipo} para el tipo de red de
      la tarjeta.
\item Se necesita llevar un registro de los empleados por lo que
      creamos una entidad \textbf{Empleado} y consideramos como
      llave primaria a un identificador único \textbf{IDEmpleado}.
      Sus atributos son \textbf{Fecha} para la fecha de nacimiento,
      \textbf{Sexo}, \textbf{Estudios} para el grado máximo de estudios y
      hicimos dos atributos compustos \textbf{Nombre} y \textbf{Dirección},
      para \textbf{Nombre} consideramos \textbf{Nombre},
      \textbf{Paterno} y \textbf{Materno}, y para \textbf{Dirección}
      consideramos \textbf{Número}, \textbf{Calle}, \textbf{Colonia},
      \textbf{Alcaldía} y \textbf{CP}.
      \item Creamos la entidad  \textbf{Licencias} para registrar las licencias de conducir. Usamos el atributo  \textbf{IDLicencia} como identificador de la entidad e incluimos atributos que creemos nos podrían interesar sobre una licencia que son:  \textbf{Tipo},  \textbf{FechaExpedición} y  \textbf{TipoSangre} de la persona a la que pertenece la licencia.
      \item Puesto que se realiza un examen médico cada 6 meses los empleados
      establecimos una entidad \textbf{Examen Médico}, agregamos
      una llave primaria \textbf{IDExamenMédico} y sus atributos
      son \textbf{peso}, \textbf{talla}, \textbf{presión} y \textbf{status}
      \item Para tener registro de los vehículos usamos la entidad \textbf{Vehículo} con atributos \textbf{Capacidad},  \textbf{Fecha} que comenzó a operar, \textbf{Combustible},  \textbf{EnOperación} y  \textbf{Placas}. El atributo  \textbf{EnOperación} nos indica si el vehículo está operando normalmente o no pues se encuentra en reparación. Incluimos el atributo  \textbf{Placas} para utilizarlo como identificador único de vehículo.
      \item Para los vehículos que se encuentran en reparación creamos la entidad \textbf{Taller} con atributos \textbf{Nombre}, \textbf{Dirección} y un identificador único \textbf{IDTaller}.
      \end{itemize}

      \textbf{- Relaciones: }

      \begin{itemize}
      \item Relacionamos Vehículo con Empleado  con la relación
            \textbf{Operar}, como un vehículo puede ser operado por muchos
            empleados pero un empleado solo puede operar un vehículo
            la cardinalidad es $1:N$, su participación es total de ambos
            lados debido a que un empleado debe operar un vehículo y un
            vehículo debe ser operado por al menos un empleado.
      \item Para relacionar SIM con Vehículo hicimos
            una relación \textbf{Asignar,} como una sola tarjeta SIM
            se asigna a un solo vehículo la cardinalidad es $1:1$, su
            participación es total del lado de Vehículo y
            parcial del lado de SIM debido a que puede haber tarjetas SIM
            sin ser asignadas pero cada carro debe contar con una tarjeta
            SIM. Además agregamos los atributos \textbf{fecha} y \textbf{hora}
            ya que sólo si la tarjeta se encuentra asignada queremos saber
            la fecha y hora que fue asignada.
      \item Se relaciona Vehículo con Taller por medio de la relación \textbf{Reparar},  como un taller puede reparar varios vehículos pero un vehículo solo puede estar siendo reparado por un solo taller la cardinalidad de la relación es $N:1$. Como nos interesa saber cuánto tiempo ha estado el vehículo fuera de servicio y las razones  para ser reparado la relación tiene como atributos la \textbf{FechaIngreso} que es cuando el vehículo deja de operar e ingresa al taller, la \textbf{FechaSalida} que es la fecha estimada en la que saldrá del taller y el atributo \textbf{Razón} que es la razón por la que debe hacerse la reparación.
      \item Relacionamos el Examen Médico con Empleado con la relación
            \textbf{Realizar}, como un empleado puede tener muchos
            examenes médicos la cardinalidad es $1:N$, su participación
            es total de ambos lados ya que a todos los empleados se les
            realiza un examen médico y un examen médico solo se realiza
            si hay un empleado a cual hacérselo. También agregamos los
            atributos \textbf{Fecha}, \textbf{Hora} y \textbf{Médico}
            debido a que puede ser realizado en diferentes fechas y el médico
            encargado puede ser diferente por lo que se almacenara su cédula.
      \item Relacionamos Empleado con Licencia a partir de la relación
            \textbf{Tener}, como un empleado puede tener varias licencias
            pero una licencia solo puede relacionarse con un empleado
            la cardinalidad es $1:N$, su participación es total de
            ambos lados ya que un empleado debe tener al menos una licencia
            y cada licencia debe de estar relacionado con un empleado. agregamos
            el atributo \textbf{FechaVencimiento} para tener un historial de
            las licencias del empleado.
      \end{itemize}

\section{Bitácora}
\begin{itemize}
    \item \textbf{Inicio del proyecto.}

    Lunes 02 de Septiembre, 2019.\\
    En nuestra primera reunión de equipo para este proyecto, empezamos por dar una lluvia de ideas para ir traduciendo el caso de uso a un diagrama, hicmmos anotaciones en cuaderno e hicimos algunas preguntas al ayudante acerca de la estructura, planteamos relaciones y entidades básicas, es decir, un bosquejo sin especificar completamente, hicimos por otro lado, investigaciones acerca de los tipos de Licencia y vehículos.

    \item \textbf{Desarrollo del proyecto.}

    Miercóles 04 de Septiembre, 2019.\\
    Dado lo que habíamos hecho en el laboratorio y nuestros bosquejos en cuaderno, empezamos a construir nuestro diagrama en draw.io , empezamos a asignar la cardinalidad, los atributos (compuestos en caso de que así lo requirieran), llaves y participación al diagrama.


    \item \textbf{Últimos detalles del proyecto.}

    Sábado 07 de Septiembre, 2019.\\
    Para este día ya no nos reunimos, sólo aclaramos dudas e hicimos modificaciones pertinentes al diagramad mediante github, nos encargamos de revisar la estructura del diagrama y modificar pequeños detalles como mejorar la vista o los nombres de los atributos.

    \item \textbf{Finalización del proyecto.}

    Lunes 09 de Septiembre, 2019.\\
    El día de hoy sólo discutimos la realización de la bitácora y la estructura de la misma, hicimos algunas modificaciones finales y completamos el diagrama en su totalidad.

\end{itemize}

\end{document}
