\documentclass[letterpaper,11pt]{article}
% Soporte para los acentos.
\usepackage[utf8]{inputenc}
\usepackage[T1]{fontenc}
% Idioma español.
\usepackage[spanish,mexico, es-tabla]{babel}
% Soporte de símbolos adicionales (matemáticas)
\usepackage{multirow}
\usepackage{amsmath}
\usepackage{amssymb}
\usepackage{amsthm}
\usepackage{amsfonts}
\usepackage{latexsym}
\usepackage{enumerate}
\usepackage{ragged2e}
\usepackage{mathtools}
\usepackage{float}
% Soporte para referencias y citas
\usepackage{hyperref}
% Soporte para imágenes.
\usepackage{graphicx}
% Modificamos los márgenes del documento.
\usepackage[lmargin=2cm,rmargin=2cm,top=2cm,bottom=2cm]{geometry}

\title{Fundamentos de Bases de Datos \\
       Práctica 03. }

\author{Teresa Becerril Torres
        $\#$ de cuenta: $315045132$ \\
        Miguel Ángel Torres Sánchez
        $\#$ de cuenta: $315300442$ \\
        Nicole Romina Traschikoff García
        $\#$ de cuenta: $315164482$ \\
        Tania Michelle Rubí Rojas
        $\#$ de cuenta: $315121719$}

\date{09 de septiembre del 2019}
\begin{document}
\maketitle

\section{Bitácora}
\begin{itemize}
    \item \textbf{Inicio del proyecto.}

    Lunes 02 de Septiembre, 2019.\\
    En nuestra primera reunión de equipo para este proyecto, empezamos por dar una lluvia de ideas para ir traduciendo el caso de uso a un diagrama, hicmmos anotaciones en cuaderno e hicimos algunas preguntas al ayudante acerca de la estructura, planteamos relaciones y entidades básicas, es decir, un bosquejo sin especificar completamente, hicimos por otro lado, investigaciones acerca de los tipos de Licencia y vehículos.

    \item \textbf{Desarrollo del proyecto.}

    Miercóles 04 de Septiembre, 2019.\\
    Dado lo que habíamos hecho en el laboratorio y nuestros bosquejos en cuaderno, empezamos a construir nuestro diagrama en draw.io , empezamos a asignar la cardinalidad, los atributos (compuestos en caso de que así lo requirieran), llaves y participación al diagrama.


    \item \textbf{Últimos detalles del proyecto.}

    Sábado 07 de Septiembre, 2019.\\
    Para este día ya no nos reunimos, sólo aclaramos dudas e hicimos modificaciones pertinentes al diagramad mediante github, nos encargamos de revisar la estructura del diagrama y modificar pequeños detalles como mejorar la vista o los nombres de los atributos.

    \item \textbf{Finalización del proyecto.}
    
    Lunes 09 de Septiembre, 2019.\\
    El día de hoy sólo discutimos la realización de la bitácora y la estructura de la misma, hicimos algunas modificaciones finales y completamos el diagrama en su totalidad.

\end{itemize}

\end{document}
